\documentclass{jlreq}
\usepackage{graphicx}
\usepackage{float}
\usepackage{xcolor}
\usepackage{indentfirst}
\usepackage{fancybox, ascmac}
\usepackage{listings}
\usepackage{url}
\lstset{
    basicstyle=\ttfamily\small,
    %numbers=left,
    keywordstyle=\color{blue},
    commentstyle=\color{green!50!black},
    stringstyle=\color{orange},
    frame=single,
    tabsize=4,
    breaklines=true
}
\title{wa}
\author{学籍番号~1270352\\
        中西~颯汰\\
        グループ~05\\
        }
\date{\today}
\begin{document}
\maketitle
\clearpage
\tableofcontents
\clearpage
\section{概要}
\section{目的}
今回の実験の主な目的は、目標刺激と妨害刺激の組み合わせによって探索非対称性が生じるかを検証することである。
探索非対称性とは、目標刺激と妨害刺激を入れ替えると、探索時間が大きく異なる現象を指す。
この現象の有無は、セットサイズ(刺激の数)の変化が探索時間(反応時間)に与える影響が異なるかどうかで判断する。
回帰分析を用いて回帰直線の傾きが目標刺激の種類によって異なるかを統計的に分析する 。
\section{方法}
\subsection{1回目}
\subsubsection*{実験プログラム}
\subsubsection*{課題}
\subsubsection*{データ処理・分析}

\subsection{2回目}
\subsubsection*{データまとめ}
\subsubsection*{回帰分析}
\subsubsection*{統計的仮説検定}

\section{結果}
\subsection{1回目}
\subsection{2回目}
\section{考察}
\begin{thebibliography}{9}
\bibitem{page1}

\url{https://jp.mathworks.com/help/}
\end{thebibliography}
\end{document}

