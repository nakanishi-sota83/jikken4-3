\documentclass{jlreq}
\usepackage{graphicx}
\usepackage{float}
\usepackage{xcolor}
\usepackage{indentfirst}
\usepackage{fancybox, ascmac}
\usepackage{listings}
\usepackage{url}
\lstset{
    basicstyle=\ttfamily\small,
    %numbers=left,
    keywordstyle=\color{blue},
    commentstyle=\color{green!50!black},
    stringstyle=\color{orange},
    frame=single,
    tabsize=4,
    breaklines=true
}
\title{wa}
\author{学籍番号~1270352\\
        中西~颯汰\\
        グループ~05\\
        }
\date{\today}
\begin{document}
\maketitle
\clearpage
\tableofcontents
\clearpage
\section{概要}
本実験では、複数の妨害刺激の中から目標刺激を探し出す視覚探索課題を用いて、視覚処理のメカニズムを検討した。
視覚探索にはポップアウトなどの並列処理と、注意を必要とする直列処理の二つのモードがある。



\section{目的}
\section{方法}
\section{結果}
\section{考察}
\section{結論}
\begin{thebibliography}{9}
\bibitem{page1}

\url{}
\end{thebibliography}
\end{document}

