\documentclass{jlreq}
\usepackage{graphicx}
\usepackage{float}
\usepackage{xcolor}
\usepackage{indentfirst}
\usepackage{fancybox, ascmac}
\usepackage{listings}
\usepackage{url}
\lstset{
    basicstyle=\ttfamily\small,
    %numbers=left,
    keywordstyle=\color{blue},
    commentstyle=\color{green!50!black},
    stringstyle=\color{orange},
    frame=single,
    tabsize=4,
    breaklines=true
}
\title{心理物理学による視覚探索の分析}
\author{学籍番号~1270352\\
        中西~颯汰\\
        グループ~05\\
        }
\date{\today}
\begin{document}
\maketitle
\clearpage 
\tableofcontents
\clearpage
\section{概要}
本実験は、心理物理学の一環として視覚探索課題における人間の情報処理メカニズムを検討することを目的とする。
心理物理学とは、刺激(入力)と知覚・反応(出力)の関数関係を調べることで、人間の内的な情報処理のメカニズムを研究する学問分野である。
今回は、複数の妨害刺激(distractor)の中から目標刺激(target)を見つけ出す「視覚探索」課題を用いて、反応時間(刺激が提示されてから反応が生起するまでの所要時間)を測定・分析することで、視覚系の処理メカニズムについて検討した。

\section{目的}
視覚探索課題において、目標刺激の有無や種類、妨害刺激の数(セットサイズ)が反応時間に与える影響を測定し、並列探索と直列探索という二つの探索モードの特性を明らかにすることを目的とした。

1日目に実施した実験で得られた複数の参加者のデータを分析し、目標刺激と妨害刺激を入れ替えた際に探索時間が大きく異なる「探索非対称性」が生じるかを検証する。回帰分析を用いて、セットサイズと反応時間の関係を示す回帰直線の傾きを算出し、その傾きが目標刺激の種類によって統計的に有意な差があるかを、対応のあるt検定により分析することを目的とした。
\section{1回目の方法}
\subsection{課題1:視覚探索実験の実施}
刺激は、目標刺激が0個または1個のいずれかであった。
目標刺激の種類は2種類あり、目標刺激が「○」の時は妨害刺激が「○」に「棒」を足したもの、
目標刺激が「○」に「棒」を足したものの時は妨害刺激が「○」であった。
妨害刺激の数、すなわちセットサイズは4, 8, 16のいずれかで、試行ごとにランダムに変わるように設定された。

実験プログラムvisualsearch.mをMatlabで開き、自身の学籍番号を入力して実験を開始した。
ディスプレイまでの観察距離は約40cmに設定され、試行ごとに顔の位置が変動しないよう指示された(視角の統制のため)。
なにかキーを押すと注視点が0.5秒間提示され、続いて刺激が呈示された。
参加者には、目標刺激があれば「F」キーを、なければ「J」キーを、できるだけ早く正確に押すよう教示された。
実験は、最初に練習試行12試行、続いて本実験240試行から構成された。
反応時間が5秒以上かかった試行(時間オーバー)や不正解の試行は、データ分析から除外された。

\subsection{課題2}
\subsubsection{データのインポートと前処理}
実験後にMatlabで出力されたdata1270352.txtファイルは、Matlabのimportdata関数を用いてインポートした。
インポート後、練習試行のデータと正解以外のデータ(誤答、違うキー、時間切れ)は、分析対象から除外した。
\subsubsection{条件の分類}
今回の視覚探索の結果は以下のように出力される。
\begin{figure}[H]
    \centering
    \includegraphics[width=0.6\linewidth]{day1/kekka.jpg}
    \caption{結果の見え方}
    \label{fig:kadai1}
\end{figure}
取得したデータの各行は、$participant$, $practice/real$, $trial\#$, $target/nontarget$, $target type$, $setsize$, $error$, $responsetime$の8つの列から構成されている。
このうち、$target/nontarget$(4列目)と$target type$(5列目)を用いて、以下の4つの条件にデータを論理的に分類した。
cond1: 目標刺激あり(data(:,4)==1)で、目標刺激の種類が「○」(data(:,5)==1)のデータ

cond2: 目標刺激あり(data(:,4)==1)で、目標刺激の種類が「棒」(data(:,5)==2)のデータ

cond3: 目標刺激なし(data(:,4)==2)で、目標刺激の種類が「○」(data(:,5)==1)のデータ

cond4: 目標刺激なし(data(:,4)==2)で、目標刺激の種類が「棒」(data(:,5)==2)のデータ
\subsubsection{平均反応時間の算出}
セットサイズ(data(:,6))のユニークな値(4, 8, 16)を取得し、各条件と各セットサイズの組み合わせについて、反応時間(data(:,8))の平均値をmean関数を用いて算出した。
\subsubsection{グラフの作成}
算出した平均値を用いて、セットサイズと平均反応時間の関係を示す折れ線グラフを作成した。
Matlabのplot関数を用いて、各条件を異なる線の種類とマーカーで表現した。具体的には、目標刺激ありの条件は「○」マーカー、目標刺激なしの条件は「$\^$」マーカーを使用し、目標刺激の種類に応じて実線または破線で区別した。
これにより、セットサイズと反応時間の関係を視覚的に比較できるようにした。



以下のグラフがこの課題2の全体コードである。
\begin{lstlisting}
A = importdata('data1270352本物.txt', ',' , 1);
data = A.data;

% 条件の論理インデックス
cond1 = (data(:,4)==1) & (data(:,5)==1);
cond2 = (data(:,4)==1) & (data(:,5)==2);
cond3 = (data(:,4)==2) & (data(:,5)==1);
cond4 = (data(:,4)==2) & (data(:,5)==2);

% セットサイズのユニーク値を取得(昇順に)
set_sizes = unique(data(:,6));

% 平均反応時間を格納する配列
mean_rt1 = zeros(size(set_sizes));
mean_rt2 = zeros(size(set_sizes));
mean_rt3 = zeros(size(set_sizes));
mean_rt4 = zeros(size(set_sizes));

% 各セットサイズでの平均値を計算
for i = 1:length(set_sizes)
    ss = set_sizes(i);
    mean_rt1(i) = mean(data(cond1 & data(:,6)==ss, 8));
    mean_rt2(i) = mean(data(cond2 & data(:,6)==ss, 8));
    mean_rt3(i) = mean(data(cond3 & data(:,6)==ss, 8));
    mean_rt4(i) = mean(data(cond4 & data(:,6)==ss, 8));
end

% 折れ線グラフの描画
figure;
plot(set_sizes, mean_rt1, '-o', 'LineWidth', 2); hold on; % ○実線(あり・丸)
plot(set_sizes, mean_rt2, '--o', 'LineWidth', 2);         % ○破線(あり・丸+棒)
plot(set_sizes, mean_rt3, '-^', 'LineWidth', 2);          % △実線(なし・丸)
plot(set_sizes, mean_rt4, '--^', 'LineWidth', 2);

hold off;

xlabel('セットサイズ');
ylabel('平均反応時間 [ms]');
title('条件別 平均反応時間の折れ線グラフ');
xticks([4 8 16]);
xlim([3 17]);
legend({'目標刺激が丸', '目標刺激が丸+棒', '目標刺激がなし・丸', '目標刺激がなし・丸+棒'}, 'Location', 'northwest');
grid on;
\end{lstlisting}
\subsection{課題3:実験プログラムの改変}
探索非対称性を生じさせると考えられる刺激として、目標刺激と妨害刺激の組み合わせを「○」と「C」に変更することをグループで検討し、プログラムを改変した。

元のプログラムでは、円と円に線を加えた刺激を描画していたが、改変にあたり、「C」の刺激をPsychtoolboxのScreen関数を用いて以下のように描画した。

円の描画: Screen('FrameOval', circle, black, ovalRect, stroke); は、オフスクリーンウィンドウcircleに、
黒い線で指定された矩形ovalRect内に円を描画する。この円は目標刺激「○」として用いられる。


「C」の描画: 「C」の刺激を作成するために、Screen('FillRect', cplusline, black, idou\_lineRect);  の箇所を修正した。
具体的には、円を描画した後に、その一部を背景色(gray)で塗りつぶすことで、円が途切れた「C」の形状を表現した。
これにより、目標刺激「C」が完成する。
\begin{figure}[H]
    \centering
    \includegraphics[width=0.6\linewidth]{day1/sample.jpg}
    \caption{Cの見た目}
    \label{fig:kadai1}
\end{figure}
プログラムのロジック: whichtargetという変数で目標刺激の種類を制御し、whichtarget == 1の場合に目標刺激を「C」、whichtarget == 2の場合に目標刺激を「○」とした。
また、それぞれの妨害刺激は、目標刺激とは異なる刺激(例:目標が「C」なら妨害が「○」)となるように設定された。

改変したプログラムを用いて再度実験を行い、目標刺激が「○」の場合と「C」の場合で、妨害刺激の数に対する反応時間の増加率に差が見られるかを確認し、探索非対称性の有無を検証した。


\section{2回目の方法}
\subsection{課題1}
1日目に各自で実施した実験のデータを複数人分まとめ、分析を行った。
まず、前回算出した各自の反応時間の平均値のデータをGoogleスプレッドシートを用いて集計した。
この際、データはミリ秒単位であり、データ取得の精度を考慮して小数点以下を四捨五入し、どの条件のデータかを間違えないように慎重に記入した。

次に、この集計したデータから、各条件の反応時間の平均値と標準誤差を算出した。
平均値の算出にはExcelのAVERAGE関数を用い、標準誤差を求めるために、まずVAR.S関数で不偏分散を求め、その平方根である標準偏差を計算した。
標準誤差は、標準偏差をサンプルサイズ(n)の平方根で割ることで求められ、サンプルサイズはCOUNT関数を用いて数えた。
標準誤差は、もし同じ条件で何度も標本を取った場合に、標本平均がどれくらいばらつくかを示す値である。

最後に、複数人の参加者データの平均値について、1日目に作成したようなグラフにまとめた。
さらに、算出した標準誤差をグラフに誤差線として表示した。
Excelでは、グラフを選択後、「グラフのデザイン」→「グラフ要素を追加」→「誤差範囲」から設定を行い、「その他の誤差範囲オプション」で算出した標準誤差の値を指定することで、誤差線を表示した。
これにより、データのばらつきを視覚的に捉えることが可能になった。
\subsection{課題2}
探索非対称性が生じているかを検証するため、回帰分析とt検定を用いて定量的な分析を行った。

まず、回帰分析によって、2種類の目標刺激の条件それぞれについて、セットサイズと平均反応時間の関係を示す回帰直線の傾きを求めた。
この傾きは、セットサイズが1つ増加した際に反応時間が平均してどれだけ増加するかを示す回帰係数である。
Excelを用いる場合は、グラフの近似曲線オプションから数式を表示させるか、SLOPE関数を用いることで算出できる。


次に、各参加者について、2つの条件(目標刺激が「○」と「+」)における回帰直線の傾きの差を求めた。
この傾きの差のデータセットを用いて、参加者間の平均値と標準誤差を算出した。

最後に、この傾きの差の平均値と標準誤差を用いてt値を算出し、統計的仮説検定を行った。
今回は、同一参加者が2つの異なる条件を経験しているため、対応のあるt検定を適用した。
この検定の帰無仮説は、「目標刺激が2種類の場合で、回帰直線の傾きの平均値に差がない」というものであり、対立仮説は「傾きの平均値に差がある」というものである。
t値と自由度(サンプルサイズ-1)から有意確率を求めた。
有意確率の算出には、ExcelのTDIST関数やMatlabのtcdf関数を用いた。
検算として、ExcelのTTEST関数やMatlabのttest関数を用いて、有意水準5\%(alpha=0.05)で帰無仮説を棄却できるかを確認した。
これにより、探索非対称性が統計的に有意な現象であるかを検証した。
\section{結果}
\subsection{1回目}
今回行った実験の結果は、図3のグラフに示す通りである。
このグラフは、個人の実験データに基づいて作成したもので、セットサイズと平均反応時間の関係を示している。

目標刺激ありの条件: 目標刺激が「○」の場合(青色の実線)と「○」に「棒」を加えたもの(橙色の実線)では、セットサイズが大きくなるにつれて平均反応時間がわずかに増加する傾向が見られた。
これは、刺激数が増加するにつれて探索に時間を要する、直列探索の特性を示唆している。

目標刺激なしの条件: 目標刺激がなかった場合、「○」の妨害刺激のみの条件(黄色の実線)と、「○」と「棒」を加えた妨害刺激のみの条件(紫色の破線)のいずれにおいても、セットサイズが増加するにつれて反応時間がわずかに増加している。
これは、探索を完了するために全ての刺激を順次確認する必要があることを示している。

\begin{figure}[H]
    \centering
    \includegraphics[width=0.6\linewidth]{day1/gurafu.png}
    \caption{条件別反応時間の折れ線グラフ}
    \label{fig:kadai1}
\end{figure}
\begin{figure}[H]
    \centering
    \includegraphics[width=0.6\linewidth]{day1/excel_gurafu.jpg}
    \caption{Excelで作った折れ線グラフ}
    \label{fig:kadai1}
\end{figure}
\begin{figure}[H]
    \centering
    \includegraphics[width=0.6\linewidth]{day1/excel_hyou.jpg}
    \caption{Excelでの図を作るために用いた表}
    \label{fig:kadai1}
\end{figure}
\subsection{2回目}
複数の参加者のデータを集計し、分析した結果を図6に示す 。このグラフは、各条件の平均反応時間に標準誤差の誤差線を加えたものである。


目標刺激ありの条件: 「12平均」として示されている条件(目標刺激が「○+棒」の妨害刺激から「○」を探す条件)では、セットサイズが増加するにつれて平均反応時間が大きく増加する傾向が見られた。
一方、「11平均」として示されている条件(目標刺激が「○」の妨害刺激から「○+棒」を探す条件)では、平均反応時間の増加が比較的緩やかであった。

目標刺激なしの条件: 目標刺激がなかった場合(「21平均」と「22平均」)、いずれの条件でも反応時間はセットサイズにあまり依存せず、ほぼ一定の値を示した。
この結果は、目標刺激がない場合に比べ、目標刺激がある場合のほうが探索にかかる時間の増加率が大きいことを示している。
\begin{figure}[H]
    \centering
    \includegraphics[width=0.6\linewidth]{day2/gurafu.png}
    \caption{標準誤差の誤差線をつけた平均のグラフ}
    \label{fig:kadai1}
\end{figure}
\begin{figure}[H]
    \centering
    \includegraphics[width=0.6\linewidth]{day2/excel.jpg}
    \caption{グラフを作成するために用いた表}
    \label{fig:kadai1}
\end{figure}
\section{考察}
本実験の結果は、視覚探索における直列探索と並列探索の特性、および探索非対称性の存在を支持するものであった。
まず、個人の結果(図3)および複数人分の集計結果(図6)において、セットサイズが増加するにつれて反応時間が増加する傾向が確認された。
これは、目標刺激を順次処理する直列探索の特性を示している。

一方、一部の条件では、反応時間の増加が比較的緩やかであった。
これは、目標刺激が妨害刺激の中から瞬時に「ポップアウト」し、並列に処理されていることを示唆している。

最も重要な発見は、探索非対称性が確認された点である。
複数の参加者データを集計した結果(図6)、目標刺激と妨害刺激の組み合わせを入れ替えると、回帰直線の傾きが大きく異なっていた。
この現象は、トリーズマンの提唱する理論、すなわち「標準的特徴」とそこからの「逸脱」に基づく情報処理メカニズムによって説明できる。

本研究の目的は、視覚探索における探索モードと探索非対称性を明らかにすることであり、結果の分析からその目的は十分に達成されたと結論づけることができる。
しかし、1日目の課題3では、目標刺激を「○」と「C」に変更して探索非対称性を検証しようとしたが、データが集められなかった。
この不備により、異なる刺激条件下での検証が不十分なままに終わってしまった。
今後の課題として、この点についてプログラムや手続きを改善し、再度実験を行う必要があるだろう。
\section{結論}
本実験は、視覚探索課題における人間の情報処理メカニズムを検討したものであり、視覚探索にはセットサイズに比例して探索時間が増加する直列探索と、セットサイズにほとんど影響されない並列探索という二つのモードが存在することが確認された。
また、目標刺激と妨害刺激の組み合わせによって探索時間が大きく異なる探索非対称性が確認された。
この探索非対称性は、視覚系が持つ「標準的特徴」とそこからの「逸脱」に基づく情報処理メカニズムによって説明できる可能性が高い。
これらの結果は、人間の視覚が刺激の物理的特性に基づいて柔軟かつ効率的に情報を処理していることを示唆している。

\begin{thebibliography}{9}
\bibitem{page1}
情報学群実験第4~講義資料 ~~~~~繁桝博昭

\bibitem{page2}
Psychtoolbox 公式ウェブサイト
\url{http://psychtoolbox.org/}
\end{thebibliography}
\end{document}

